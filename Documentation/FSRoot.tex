\documentclass[11pt]{article}
%\usepackage{graphicx}
\usepackage{amssymb}
%\usepackage[]{epsfig}
%\usepackage{parskip}

\textheight 8.0in
\topmargin 0.0in
\textwidth 6.0in
\oddsidemargin 0.25in
\evensidemargin 0.25in

\newcommand{\FSR}{{\tt FSRoot}}
\newcommand{\git}{{\tt git}}

\begin{document}


\title{Notes on the \FSR\ Package}
\author{Ryan Mitchell}
\date{\today}
\maketitle

\abstract{
\FSR\ is a set of utilities to help manipulate information about different Final States~(FS) produced in particle physics experiments.  The utilities are built around the CERN {\tt ROOT} framework.  This document provides an introduction to \FSR.}

\tableofcontents

\parindent 0pt
\parskip 10pt


\section{Overview of \FSR}

This is an overview of \FSR.

\section{Download and Setup}

There are three initial steps:

(1) The macros can be downloaded from the \git\ repository.  This command will download the macros into a directory called \FSR:
\begin{verbatim}
> git clone <user>@stanley.physics.indiana.edu:/home/s4/remitche/git/RootMacros.git
\end{verbatim}

(2) The {\tt \$ROOTMACROS} environment variable needs to be set to point to the \FSR\ directory.  For example, in {\tt tcsh} (usually using the {\tt .tcshrc} file):
\begin{verbatim}
> setenv ROOTMACROS <RootMacros directory>
\end{verbatim}

(3) The following lines need to be added to a file called {\tt .rootrc} (usually in your home directory):
\begin{verbatim}
Unix.*.Root.DynamicPath: .:$(ROOTMACROS):$(ROOTSYS)/lib:
Unix.*.Root.MacroPath:   .:$(ROOTMACROS):
\end{verbatim}

\section{Getting Updates}

To get updates, do this from the {\tt \$ROOTMACROS} directory:
\begin{verbatim}
> git pull
\end{verbatim}

\end{document}

\documentclass[11pt]{article}
%\usepackage{graphicx}
\usepackage{amssymb}
%\usepackage[]{epsfig}
%\usepackage{parskip}

\textheight 8.0in
\topmargin 0.0in
\textwidth 6.0in
\oddsidemargin 0.25in
\evensidemargin 0.25in

\newcommand{\FSR}{{\tt FSRoot}}
\newcommand{\git}{{\tt git}}

\begin{document}


\title{Notes on the \FSR\ Package}
\author{Ryan Mitchell}
\date{\today}
\maketitle

\abstract{
\FSR\ is a set of utilities to help manipulate information about different Final States~(FS) produced in particle physics experiments.  The utilities are built around the CERN {\tt ROOT} framework.  This document provides an introduction to \FSR.}

\tableofcontents

\parindent 0pt
\parskip 10pt



%\section{Overview of \FSR}

%This is an overview of \FSR.

\section{Installation and Initial Setup}

Instructions for installation and initial setup:

(1) Download the source:
\begin{verbatim}
    git clone https://github.com/remitche66/FSRoot.git FSRoot
\end{verbatim}

(2) Set the location of \FSR\ in your login shell script (e.g. {\tt .cshrc}):
\begin{verbatim}
    setenv FSROOT [xxxxx]/FSRoot
\end{verbatim}

(3) Also probably add the \FSR\ directory to {\tt \$DYLD\_LIBRARY\_PATH} and {\tt \$LD\_LIBRARY\_PATH}.  This allows you to compile code including \FSR\ functions.  For example:
\begin{verbatim}
     setenv DYLD_LIBRARY_PATH $DYLD_LIBRARY_PATH\:$FSROOT
     setenv   LD_LIBRARY_PATH   $LD_LIBRARY_PATH\:$FSROOT
\end{verbatim}

(4) There is usually a {\tt .rootrc} file in your home directory that {\tt ROOT} uses for initialization.  Add lines like these to {\tt .rootrc}, which tell {\tt ROOT} the location of {\tt FSRoot}:
\begin{verbatim}
    Unix.*.Root.DynamicPath: .:$(FSROOT):$(ROOTSYS)/lib:
    Unix.*.Root.MacroPath:   .:$(FSROOT):
\end{verbatim}

(5) Now when you open {\tt ROOT}, the \FSR\ utilities should be loaded and compiled -- you should see a message saying "Loading the FSRoot Macros" along with the output of the compilation.


\end{document}
